\documentclass[12pt]{beamer}
%\documentclass[12pt,draft]{beamer} %Para dejar solo el contenido
\usetheme{Berlin}
%===================Temas predefinidos de colores=========================
%\usecolortheme{albatross}
%\usecolortheme{beaver}
%\usecolortheme{beetle}
%\usecolortheme{crane}
%===================Ajustes personalizados de colores ====================
\usepackage{xcolor}%Para ajustes sobre colores.  
\definecolor{Color1}{RGB}{0,200,0}
\definecolor{Color2}{RGB}{200,0,200}
\usepackage{comment}
\begin{comment}
%\setbeamercolor{title}{fg=Color1}
%Para ajustar el color del titulo, el primer plano se define a través de fg (foreground) y el fondo se define por bg (background)
\setbeamercolor{title}{fg=Color1, bg=Color2}% Ajuste del título.
%Ajuste de los títulos de los frames.
\setbeamercolor{frametitle}{bg=Color1, fg=Color2}
%Ajuste de la estructura restante del documento. 
\setbeamercolor{structure}{fg=yellow}
%Ajuste del color del texto de todas la presentación.
\setbeamercolor{normal text}{fg=blue,bg=cyan}
\end{comment}
%==========================Ajustes del tipo de letra===================== 
\usefonttheme{serif}
%Lista de opciones: default, serif, structurebold, structureitalicserif, structuresmallcapsserif, professionalfonts, serifbold, serifitalic
\usepackage[utf8]{inputenc}
\usepackage[spanish]{babel}
\usepackage{amsmath}
\usepackage{amsfonts}
\usepackage{amssymb}
\usepackage{amsthm}%Paquete para teoremas
\usepackage{graphicx}
\usepackage{tikz}
%================Personalizando Comandos==================================
\newcommand{\norma}[1]{\lVert #1 \rVert}
\newcommand{\norm}[1]{\left \lVert #1 \right \rVert}
\newcommand{\br}[1]{\left \lbrace #1 \right\rbrace}
\newcommand{\mb}[2]{\mathbb{#1}^#2}
%================Definiendo Teoremas======================================
\setbeamertemplate{theorems}[numbered]%Numerara teoremas
\newtheorem{Teo}{\textbf{Teorema}}
\newtheorem{Def}{\textbf{Definición}}[section]
\newtheorem{Prop}{\textbf{Proposición}}[section]
%================Ajustes de video========================================
\usepackage{animate}
%ffmpeg -i file_example_MP4_480_1_5MG.mp4 Frames_%04.png
\usepackage{multimedia}
%================Titulo de la presentacion===============================
\title{Documento Beamer para la clase de MM116
}
\author{\inst{1}Jose \and \inst{2}Mateo}
%\setbeamercovered{transparent} 
%\setbeamertemplate{navigation symbols}{} 
%===============Ajuste del logo de la presentacion========================
%\logo{\includegraphics[scale=0.1]{Logo_UNAH_Gris}}
%\titlegraphic{\includegraphics[scale=0.1]{Logo_UNAH_Gris}}
\institute{\inst{1} UNAH-CU \and \inst{2} UNAH-VS} 
%\begin{comment}
\bfseries
\titlegraphic{ 
\begin{tikzpicture}[remember picture,overlay]
\node at (2.6,2.5){
    \includegraphics[scale=0.15]{Logo_UNAH_Gris}
};
\node at (-2.6,2.5){
    \includegraphics[scale=0.15]{Logo_UNAH_Gris}
};
\end{tikzpicture}
}
%\end{comment}
%\date{} 
%\subject{} 
\AtBeginSection[]{
\begin{frame}
\tableofcontents[currentsection]
\end{frame}
}
\begin{document}
\begin{frame}
\titlepage
\end{frame}
\begin{frame}{Video}
\animategraphics[autoplay,loop,width=\linewidth]{10}{Imagenes/Frames_}{0001}{0200}
\end{frame}
\section{Álgebra}
\begin{frame}{Introducción al Álgebra}
Las principales propiedades del ÁLGEBRA son:\pause
\begin{itemize}
\item<1-> Propiedad de tricotomía. 
\item<2>  Propiedad del neutro para la suma.
\end{itemize}
\onslide<3>{El teorema fundamental del algebra establece que un polinomio de grado n tiene exactamente n raices en los numeros complejos.}
Sea $p(x)$. 
\begin{Def}[Número positivo]
Un número real $x$ es positivo si $x>0$.
\end{Def}
\end{frame}
\section{Cálculo}
\begin{frame}{Introducción al Cálculo}
Las principales propiedades del ÁLGEBRA son:
\begin{itemize}
\item Propiedad de tricotomía. 
\item  Propiedad del neutro para la suma.
\end{itemize}
\end{frame}
\begin{frame}{Normas vectoriales}
Una norma $\lVert x \rVert$ ($||x||$), $\norma{x}$.\\
Algunos conjuntos conocidos $\mathbb{R}$, $\mathbb{C}$, $\mathbb{Q}$\\
Espacios vectoriales: $\mb{R}{n}$.
\end{frame}
\begin{frame}{Referencias}
\begin{thebibliography}{10}
\bibitem{Goldbach1742}[Goldbach, 1742]
Christian Goldbach.
\newblock A problem we should try to solve before the ISPN ’43 deadline,
\newblock \emph{Letter to Leonhard Euler}, 1742.
\end{thebibliography}
\end{frame}
\end{document}